%Glossaries
\makeglossaries

% You must define terms or symbols before you can use them in the document. This is best done in the preamble. Each term is defined using:
% 
% \newglossaryentry{<label>}{<settings>}
% 
% where <label> is a unique label used to identify the term. The second argument, <settings>, is a key=value comma separated list that is used to set the required information for the term. A full list of available keys can be found in "Defining Glossary Entries" in the main glossaries user manual. The principle keys are name and description.
% 
% For example, to define the term "electrolyte":
% 
% \newglossaryentry{electrolyte}{name=electrolyte,
% description={solution able to conduct electric current}}
% 
% In the above example, the label and the name happen to be the same. In the next example, the name contains a ligature but the label doesn't:
% 
% \newglossaryentry{oesophagus}{name=\oe sophagus,
% description={canal from mouth to stomach},
%plural=\oe sophagi}

%\newacronym{<label>}{<abbrv>}{<full>}

%\gls{<label>}
%\glspl{<label>}

\newacronym{}{}{}

\newacronym{sisd}{SISD}{Single Instruction, Single Data} 

\newacronym{smp}{SMP}{shared memory multiprocessor}

\newacronym{ghc}{GHC}{Glasgow Haskell Compiler}

\newacronym{ghceden}{GHC-Eden}{The Parallel Haskell Compilation System Eden}

\newacronym{gph}{GpH}{Glasgow Parallel Haskell}

\newacronym{gdh}{GdH}{Glasgow Distributed Haskell}

\newacronym{ghcsmp}{GHC-SMP}{Haskell on a Shared Memory Multiprocessor}

\newacronym{hdph}{HdpH}{Haskell Distributed Parallel Haskell}

\newacronym{ghcpps}{GHC-PPS}{GHC Parallel Profiling System}

\newacronym{pe}{PE}{Processing Element}

\newacronym{uma}{UMA}{Uniform Memory Access}

\newacronym{numa}{NUMA}{Non-Uniform Memory Access}

\newacronym{mpp}{MPP}{Massively Parallel Processors}

\newacronym{mpi}{MPI}{Message Passing Interface}

\newacronym{pvm}{PVM}{Parallel Virtual Machine}

\newacronym{hpc}{HPC}{High-Performance Computing}

\newacronym{mcnoc}{McNoC}{Multicores Network-on-Chip}

\newacronym{api}{API}{Application Programming Interface} 


%\newacronym{}{}{}



