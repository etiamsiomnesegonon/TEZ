
\chapter{Introduction}
\label{ch:introduction}

the genral theme of this thesis is to explore how changes in mass distribution can be performed to minimise the duisturbanc eto the rotational motion of a free floating object and further whether it would be possible to use as a more active control method for the rotational motion

\section{Project aims and objectives}
\label{ch:Project aims and objectives}
Current spacecrafts control their attitude in two complementary ways. One conserves angular momentum by using internal mass mouvement mostly with momentum storage devices (reaction wheels) without change to the spacecraft mass distribution. The other affects the overall angular momentum by using external torques produced mainly by thrusters which release a propellant mass. In the perspective of controlling the attitude of large uncooperative random-shape objects like asteroids these methods may prove difficult or impossible to implement. The aim of the project was to explore another way of controlling the attitude of an uncooperative random-shape object in zero-gravity by using dynamic changes in mass distribution. To this end, the main goal of this research project was to study the dynamic interactions and exchange of momentum between a randomly tumbling ellipsoid approximating a generic free-floating object and a mass moving along its surface. The system definition led to its formulation as a continuum mechanics based model and to the subsequent derivation of a simple decentralized algorithm controlling the deployment of a modular self-reconfigurable robot. The deployment of the robotic structure being more akin to an optimization problem in the sense that it should be the least disturbing to the rotational motion of the object, the project also explored a more active method of control using mass distribution through thee use of robotic arm as actuators.
\cite{ghc-pps}

\subsection{Project specific objectives}
\label{Project specific objectives}
\subsubsection{Dynamical system analysis}
\label{Dynamical system analysis}
The system was defined as the combination of the randomly tumbling object modelled as an ellipsoid and the robot deploying at its surface in constant contact with it. This system was considered as isolated and its linear translation motion neglected. The first objective of the study was to assess and quantify
\begin{enumerate}
	\item[1.]{the disturbances experienced by the initial rotational motion of the object when a spherical mass moves along its surface;}
	\item[2.]{under which conditions these disturbances can be minimised or suppressed;}
	\item[3.]{and whether these disturbances could be perceived by an appropriate sensor such as a gyroscope for instance for further use in a feedback control loop algorithm.}
\end{enumerate}
To this end a full stability study was performed as well as a search for bifurcations and an assessment of its potential chaotic behaviour.
\cite{ghc-pps}


\subsubsection{FSM algorithm and random with robotic arms}
\label{Actuation with robotic arms}
The second objective of the stydy was to design 
\cite{ghc-pps}

\subsubsection{Actuation with robotic arms}
\label{Actuation with robotic arms}
the third objective wasw to design and evaklute a more active approach by diverting robotic arm from their usula purpose to become actuators

\cite{ghc-pps}



%\subsection{Subsection}
%
%\subsubsection{Subsubsection}
%
%\cite{ghc-pps}

%Write.. \gls{ghc}.

%Write..\gls{ghc}.


\section{Introduction and motivation}
\paragraph{}	On 25/11/2015, the US congress passed the U.S. Commercial Space Launch Competitiveness Act authorising U.S. citizens to explore and recover space resources for  commercial purposes (https://www.congress.gov/bill/114th-congress/house-bill/2262). This reflect the significant and strategic part of the global economy that  the space industry now represents. In 2014, it was worth 330 billion dollars worldwide. This amount had doubled over the ten year between 2005 and 2014 growing at an average 7 percents compound annual rate. Three quarters of this output was the result of commercial activities and revenues from commercial space products and services accounted for more than a third of the global industry's worth (http://www.spacefoundation.org/media/press-releases/space-foundation-report-reveals-global-space-economy-climb-330-billion). 
\paragraph{}	 Mining space resources borne by near-Earth asteroids (NEA) will offer substantial impact and benefits: a secure access to and an increase of the supply of critical mineral resources (in high global demand and with constrained supply) providing geopolitical stability, promoting economic growth and fostering technological development in many fields like artificial intelligence and robotics ~\cite{Sommariva2015}.
\paragraph{}	Asteroids are expected to contain excellent and easy to mine ores such as precious metals like platinum group metal (PGM), gold or germanium as well as metallic industrial resources like nickel-iron and cobalt. ~\cite{Sommariva2015} Moreover, some asteroids harbour significant quantities of water which could be used in space either as fuel for space engines or for supporting life ~\cite{Elvis2014} and may harbour other volatiles such as ammonia, carbon dioxide or methane ~\cite{Sommariva2015}. The quantity of resources can only be statistically estimated ~\cite{Sanchez2012}, but precious metal are expected to be in higher concentrations in asteroids than in the Earth crust ~\cite{Sommariva2015}. In order to illustrate the economic value of these NEAs resources, it is estimated that they collectively harbour 37E15 kg of iron worth 11000 trillion dollars at current Earth prices ~\cite{LewisBook}. One 100m-diameter PMG-bearing asteroid alone would provide a tenth of the 2011 world platinum production (http://minerals.usgs.gov/minerals/pubs/commodity/platinum/mcs-2012-plati.pdf).
\paragraph{}	Most asteroids are located in the asteroids belt between Mars and Jupiter. However, it is too expansive energy-wise to reach them. Therefore, near Earth asteroids (NEAs) are considered to be mined first. These NEAs have orbits which near Earth on a regularly basis. As of 2015, the number of known NEAs is 10,337 according to the international astronomical union ~\cite{Sommariva2015}. The diameter of 861 of them is larger than one km ~\cite{Sommariva2015}. If other objects, such as comets, are considered the population of near-Earth objects (NEOs) increases significantly. For diameter larger than 100m, the number of NEOs nears 20,000 ~\cite{Mainzer2011} and for diameter larger than 20m, the number of NEOs nears 10 millions ~\cite{Mainzer2011}.
\paragraph{}	Asteroids are uncontrolled and uncooperative objects devoid of any docking device [18]. Their shapes are irregular, in particular for small ones (i.e. with diameter< 100km) and due to their their irregular shape, their rotational motion cannot be described in terms of principal axis rotation but rather as a tumbling motion i.e. a rotation around a dynamic axis ~\cite{Elvis2014,LewisBook}. For the biggest asteroids, the equatorial velocity can be of the order of tens of km/s [18]. 
\paragraph{}There have already been one successful attempt to reach asteroids. The Hayabusa mission successfully landed on the surface and collected surface sample of the asteroid Itokawa number 25143 in 2005. However, contact was brief and not secured ~\cite{LewisBook}. Secure attachment to the surface of an asteroid by a probe remains to be tested ~\cite{LewisBook}. Currently, NASA is in the planning of yet another mission: the Asteroid Redirect Mission (ARM) which proposes to bring back a captured asteroid in a lunar orbit for further human exploration ~\cite{Sommariva2015}. 
\paragraph{}Among the engineering problems faced by such missions, the attitude control problem is of particular interest for a robotic application. Attitude control constrains critical tasks among which the rendezvous and docking to the asteroid, the mining process itself and the accuracy of the trajectory of the retrieved material back to where it should be consumed whether in space or on Earth. 
\paragraph{}	Various strategies are envisaged to de-spin rotating asteroids ~\cite{Sommariva2015,LewisBook}. In ~\cite{LewisBook}, it is estimated that in order to de-spin a rapidly rotating 100m-diameter asteroid whose mass is of the order of a few million tonnes, about one tonne of chemical propellant would be required to do so completely, propellent which should be readily available on the asteroid itself. However, these methods may prove impractical or energy consuming due to the low gravity environment (of the order of mm/s2) in which secure attachment is paramount ~\cite{LewisBook}.
The most common devices controlling attitude are internal torquers like reaction wheels which are hard to scale up for large asteroids and external torquers like thrusters. Generally speaking any mass mouvement such as appendage deployments are considered as disturbances and never exploited.
\paragraph{}In this perspective, the choice of using a modular SR robot, as an alternative for asteroid manipulation or attitude control, offers several potential benefits. It proposes a control method based on mass mouvement and deployment which circumvent the risks associated with certain landing methods in micro-gravity by enveloping the asteroid. Moreover, a self-reconfigurable robot enveloping an asteroid could later provide tracks along its modules on which other robots can circulate to perform other tasks. Finally it can take advantage of its self-reconfiguration capabilities to accurately place propulsion systems to be used for material retrieval and gravity capture near the Earth. For this later case, it is important to understand that the mass payback ratio (i.e. the mass retrieved to the mass sent) is an important criteria and is one of the key parameter to be minimised during the design phase. It is with all of the above in mind that we will endeavour to propose a self-reconfigurable robotics attitude control system in this PhD study.
%\subsection{Subsection}
%\subsubsection{Subsubsection}

\section{Literature review}
\subsection{Spacecraft design basics and requirements}
\paragraph{}The space environment is hostile and unrelenting and therefore represents one of the most challenging environment for robots to operate in and one which imposes specific drivers on robotics technology. This section will be a brief introduction of spacecraft engineering concepts with an emphasis on the impact and research challenges imposed on the design of robotic systems for space.
\subsubsection{Spacecraft Design Constraints versus Robotics}
\paragraph{}Spacecraft are designed under stringent constraints a space robot has to abide by.
\paragraph{}Firstly, spacecrafts are require to have a high robustness threshold. They must be able to sustain and survive the stresses of launch and landing (in case of landing), the vacuum and radiation of space or the environment of the body it lands on. The structure and mechanical design of a robot should have the same robustness threshold ~\cite{Ellery2003}.
\paragraph{}Secondly, mechanical actuation systems are usually considered as potential single point of failure and therefore reduced to the essential while for a robot, actuation is the mode of interaction with the environment and is crucial to its performance. Spacecraft actuation systems are either propulsion systems, attitude control systems or mechanical systems for the deployment of large structures. Robotic actuation systems increase design complexity by an order of magnitude in terms of the performance of interaction tasks with the environments ~\cite{Ellery2003}.
\paragraph{}Thirdly, control in spacecraft is based on dynamic models while for robot, environments are potentially totally unknown a priori. Robots controllers need to be more adaptable and offer learning capabilities ~\cite{Ellery2003}.
\paragraph{}Usually, all spacecrafts are broken down in eight subsystems ~\cite{Ellery2003}: 
\begin{enumerate}
	\item[1.]{Propulsion system;}
	\item[2.]{Attitude control subsystem which control the orientation of the spacecraft to ensure that all components point in the correct direction;}
	\item[3.]{Structural and Mechanical;}
	\item[4.]{Power subsystem;}
	\item[5.]{Thermal control subsystem;}
	\item[6.]{Communications subsystem;}
	\item[7.]{Onboard data handling (computer) subsystem;}
	\item[8.]{Payload subsystem which provides the business end of the spacecraft.}
\end{enumerate}
\paragraph{}The five main design budgets are ~\cite{Ellery2003}: 
\begin{enumerate}
	\item[1.]{Cost budget capping the costs of the design, development, construction, validation, and launch of a spacecraft;}
	\item[2.]{Mass budget capping the total mass of the spacecraft to be launched; imposing a trade-off between lightweightness and structural flexibility; }
	\item[3.]{Propellant budget limiting the manoeuvring capability (function of the total mass of the spacecraft); }
	\item[4.]{Power budget limiting the power and energy available to each spacecraft subsystem and the payload, imposing a trade-off between power, efficiency and computational resources; }
	\item[5.]{Data budget limiting the communications capabilities and onboard storage capacity.}
\end{enumerate}
\paragraph{}Additional constraints are placed upon reliability (above 90 percents typically) which requires extensive testing and validation under space-like conditions, a capacity for upgrade and repair-by-replacement of modules disfavouring soft computing methods, and calling for limitations in mechanical complexity. These constraints impose on the design of robots for space applications the following requirements ~\cite{Ellery2003}:
\begin{enumerate}
	\item[1.]{Lightweight components minimising launch mass but resistant to launch/impact loads (eg. up to 20 g axial acceleration and 145 dB acoustic noise for launch); }
	\item[2.]{Limited volume at launch; }
	\item[3.]{Capacity to operate in vacuum environment with materials resistant to outgassing in vacuum and the use of dry lubrication; }
	\item[4.]{Control algorithms dealing with the effects of microgravity such as no ground reaction and significant non-linear dynamics effects and restricting motion with low speeds (~0.01m/s) to compensate for the lack of damping and dissipating medium.}
	\item[5.]{Capacity to sustain extreme temperature gradients and thermal cycling between -120C to 60C usually; }
	\item[6.]{Resistance of electronic components to a high charged particle radiation environment;}
	\item[7.]{Operation with limited onboard computational capabilities impacting real-time control and navigation;}
	\item[8.]{Resistance to electrostatic charging and discharges due to the lack of grounding;}
	\item[9.]{Capacity to perform image processing (if required) in a highly variable illumination environment;}
	\item[10.]{High level of autonomy.}
\end{enumerate}

\subsubsection{Whole-system Design Methodology}
\paragraph{}It is recommended that robots for space application use appropriate system engineering methods.  The entire system, comprising the robot, supporting infrastructure, the human-in-the-loop component and their interactions must be taken into account as it is more important to the success and robustness of the space-robotic system than any robot-specific technology (like mobility, dexterity or intelligence) ~\cite{Pedersen2003}.

\subsubsection{Attitude Control Hardware}
\paragraph{}This section will focus on providing some background on attitude control system already in use. In current spacecrafts, the attitude control system (ACS) provides attitude stabilization and attitude manoeuvre control. It produces control torques in response to a disturbance torques measured as an error by the attitude determination system (ADS) and in response to pointing requirements. Attitude control hardware can be divided into active and passive control devices as will be seen in the subsequent sections. The passive hardware does not consume power, does not require a communication interface and is set to remove a predefined amount of energy from the spacecraft. The active hardware can adjust the amount of energy removed. Both types of hardware modify the angular momentum of the spacecraft at any given time. 
\subsubsection{Active Attitude Control Actuators}
\paragraph{Reaction Wheels}Reaction wheels change their own angular momentum changing in turn the spacecraft angular momentum about that same axis. They are used for coarse as well as fine pointing . They can store a limited amount of angular momentum depending on power availability and motor design and experience saturation ~\cite{Lukaszynski2013}.
\paragraph{Magnetorquer}The magnetorquer is a magnetic dipole generating a moment function of the orbital position by passing through the Earth’s magnetic field. It is used in low to medium Earth orbit for coarse pointing and performs spacecraft detumbling and momentum reduction in reaction wheels. This actuator cannot be used outside earth orbit ~\cite{Lukaszynski2013}. 
\paragraph{Thrusters}Thrusters generate thrust by expelling propellant through a nozzle. When placed at a moment arm from the centre of mass of the spacecraft, they generate a torque about the centre of mass.  They work in pairs in order to minimise translational motion when used only for attitude control. They are used for both coarse and fine pointing. Their use is limited by the amount of propellant and power available available on the spacecraft ~\cite{Lukaszynski2013}.  
\paragraph{Passive Attitude Control Actuators}These actuators provide restoring torques and remove (or add sometimes) angular momentum without the need for active control. They are coarse pointing actuators only. Passive attitude actuators are not usable outside earth orbits~\cite{Lukaszynski2013}.
\paragraph{Hysteresis Rod}The hysteresis rod is a piece of ferromagnetic material which uses the variation of the magnetic field strength with the spacecraft orbital position to generate a magnetic flux density. This process reduces angular momentum of the spacecraft over time ~\cite{Lukaszynski2013}. 
\paragraph{Permanent Magnets}The permanent magnet is a permanent dipole which can only stabilize a spacecraft about two axes ~\cite{Lukaszynski2013}.

\subsubsection{Attitude Control and De-spin of an Asteroid}
\paragraph{}	This section will examine what solution are currently put forward to capture and de-spin asteroids and will focus on the very capture mechanisms and task of de-spin and not on specific mission phases as it will be the focus of our future study. These solutions assume that Near Earth Asteroids (NEAs) are tumbling, non-cooperative objects which will be de-spun autonomously in deep space. 
\paragraph{} In ~\cite{Brophy2012}, the spacecraft is intended to do some station keeping prior to capture to match the main spinning rate and axis of the NEA. Once synchronised, the asteroid would be captured in a bag where it would still have a residual relative angular velocity with the spacecraft and therefore undergo some impact. The bag would then be tightly cinched to drawn up the asteroid against the spacecraft to constrain its position and attitude so that forces and torques could be applied by the spacecraft. In order to have some order of size, the study provides an estimate of the time and mass of propellant required to de-spin the object composed of the spacecraft and the asteroid. With the hardware envisaged for the spacecraft in the study, it was estimated that in order to de-spin an asteroid of cylindrical shape of 6-m diameter x 12-m long, of mass 1,100 t and rotating at 1 RPM about its major axis, 33 minutes of continuous firing would be necessary consuming about 306 kg of propellant (one third of the total mass of the asteroid). In case the capture would not be feasible, the study finally proposes instead to anchor the spacecraft to the surface of the asteroid and winch its way to the surface in order to drill for regolith to be retrieved in the bag.
\paragraph{}In a similar study ~\cite{Roithmayr2013}, Carlos Roithmayr proposes an almost identical approach. The asteroid is also of the order of 7m diameter and has the same agreeable cylindrical shape with known moments of inertia. Again, there is no need for identification of parameters and the study confirms the order of magnitude of the propellant mass required to perform the de-spin, around 300kg for 1000t asteroid. It only optimises the moment arm length at which thruster are fired in order to maximises torques.
\paragraph{}In ~\cite{Grip2013}, again, a similar retrieving spacecraft is envisaged. However, matching the spin rate of the spacecraft with the spin rate of the asteroid along its main axis of rotation prior to capture was abandoned because of the excessive propellant expenses. The study briefly describes another enveloping approached with a bag to capturing the asteroid and facilitate passive damping of the tumbling motion toward major-axis spin only to dismiss its infeasibility on mechanical design and packaging grounds. The study then describes how the latter problem can be circumvented by proposing an inflatable exoskeleton attached directly to the spacecraft bus. The bag then collapses around the asteroid with the help of actively controlled winches to achieve passive damping. The captured asteroid is modelled by a 6-DOF joint connected to the rest of the body of the spacecraft via translational and rotational spring-dampers. Simulations seems to confirm asymptotic viscoelastic energy dissipation and convergence of the combined spacecraft-asteroid system toward a flat spin.Grip and Ono nevertheless could not conclude whether this system could be physically realizable with current space-qualified materials.
\paragraph{}Finally, in ~\cite{Shen2014} yet again, the same approach is envisaged. However, here the asteroid mass is assumed to be distributed in the most general possible way, meaning it can be unsymmetric with three distinct principal moments of inertia. Nonetheless these moments of inertia are yet again assumed to be known. As for the previous studies model-based de-spin controller are designed to be asymptotically stable, and stay within specified thruster limits, thrusting being the main actuation option. Results are again of the same order of magnitude in terms of propellant mass used.

\subsection{Active space debris capturing and removal methods}
\paragraph{}Before discussing the merit and alternatives to the methods of asteroid capture and de-spin exposed above, this section will briefly introduce concept proposed for active space debris removal, which is a similar problem in many respects, as an overview of the current trend in the industry. In ~\cite{Shan2016} a review is provided for such devices. This section essentially follows the structure of the paper.
\subsubsection{Stiff connection capturing}
\paragraph{}Capture and de-spin can be achieved by using tentacles. Tentacles embrace the space debris with a clamping mechanism either directly or from a robotic arm.  Ideally the target is embraced before physical contact. That way, the chaser satellite does not bounce and the attitude control system is allowed to stand by during capturing. The clamping mechanism then locks and the new chaser-target system turns stiff.
\paragraph{}Capture and de-spin can be effected with the use of arms. Single arm technology has been applied in many on-orbit servicing missions but always in the case of cooperative target objects. On the contrary, space debris objects are usually uncooperative and can even tumble. The three main areas considered when using arms are minimizing the impact influence, de-tumbling and attitude synchronization. The first area deals with minimising the consequences of the unavoidable impact at contact. Different methods have been suggested. For reference they are to do with controlling the direction of relative velocity between chaser and target, the  relationship between impact force and base force and the configuration between the service satellite and target. Visual servoing with a Kalman filter can also be used to predict the motion of the target with respect to chaser satellite. However, this method requires visual markers. For de-tumbling, tumbling rate below 3°/s can be captured easily. However, tumbling rate above 30°/s will not be considered. Finally, tumbling rate between 3° and 30°/s could be de-tumbled using brush contact to release the residual angular momentum of the target by soft and static contact. Further potential solutions include Ion-Beam shepherding by transfer of angular momentum and optimal control techniques with identification of the target unknown inertia parameters. Finally, attitude synchronization is an indispensable phase before capturing. For reference, nonlinear feedback control and sliding mode control could be employed for relative position tracking and attitude reorientation.
\paragraph{}Capture and de-spin can be effected with the use of multiple arms. Multiple arms have the advantages of providing a stabilizing effect as well as flexibility through cooperation.
\paragraph{}Finally, the mechanical end effector is of prime importance as it is directly involved in the capturing motion and contacts with the target. There are several concepts of mechanical effector for capturing a space debris object, such as a probe for the nozzle cone of an apogee kick motor, payload attach fitting device, articulated hand, two fingers mechanism and universal gripper.
\subsubsection{Flexible connection capturing}
\paragraph{}In the previous section, the connection between chaser satellite and target is stiff, making the combined spacecraft-object system controllable and stable. However, this solution increases mass and cost dramatically. In order to overcome this drawback, flexible connection capturing methods in which the end effector and chaser satellite are connected by a tether, are also available options.
\paragraph{}The first “flexible” option is net capturing with either a net or a gripper mechanism as end-effectors. Net capture mechanism consists of four flying weights in each corner of a net. The flying weight or bullet is shot by a spring system, the net gun. These four bullets help expand the large net to ensure that the target is wrapped up. With this method, it is not necessary to know the mass, inertia and other parameter a priori for capture. Parabolic flight experiments have been performed by GMV and ESA to validate the net deployment and capturing simulations. Net capturing is held as one of the most promising capturing methods as it allows a large distance between chaser satellite and target, so that close rendezvous and docking are not mandatory. Moreover, it is flexible, light weighted and cost efficient. However, more research is required in such areas like net modelling, contact influence, deployment and tumbling compatibility. Contact is also a problematic area as it is unavoidable during capturing process. The main risks are to create more and smaller debris and worse to lead to mission failure if the wrapping up is improper. Nonetheless net capturing  is compatible with tumbling space debris and no attitude synchronization is needed. close range rendezvous and removal would be less difficult. However, the acceptable tumbling range of a target is not yet understandable and a net may be twined by a high tumbling angular velocity thus rendering the spacecraft-object system uncontrollable.
\paragraph{}The second “flexible” option is tether–gripper mechanism. Similarly to the net capturing mechanism the end-effector in the tether–gripper mechanism is shot as a 3-finger gripper to capture a target. This 3-finger gripper is designed to be able to catch a specific part of the target precisely and stably. Requirements for tether–gripper mechanism are therefore more stringent and more complicated than net capturing. Post-capture attitude control is also problematic since the movement of the combined spacecraft-target system is unpredictable and therefore requires identification of inertia parameters to be achieved. Attitude control is a necessary condition for subsequent mission phases like de-orbiting to go ahead. 
\paragraph{}The third and last “flexible” option is a harpoon mechanism shot from the spacecraft to penetrate into large space debris objects which would be pulled to de-orbit. Despite the high risk of generating new space debris, it is considered as an attractive capturing method because of its compatibility with different shaped targets, stand-off distance allowed and no grappling point needed. However, it is not capable of dealing with a piece of debris with high tumbling rate. It is also a system favoured for anchoring a spacecraft onto an asteroid. In this latter situation, the risk of generating adverse space debris is also present. High tumbling rates would also be prohibitive.
\subsection{Conclusion and Discussion}
\paragraph{}The methods for de-spinning an asteroid described above all use model-based attitude control. These dynamic models use parameters like moment of inertia to describe the rotational motion of the spacecraft. From these models the control inputs to the various actuators are deduced. However, all of these studies assume that the asteroid dynamic parameters are known a priori and are therefore readily usable for computations. Space debris removal mission face similar problems but the above studies take the more realistic approach of identifying the unknown dynamic parameters of the target object to be de-orbited. The identification process usually relies on the input of disturbances whose measured effects feedback into the model to tune the parameters. This process is relatively inexpensive energy-wise providing the mass ratio between the spacecraft and the object is in favour of the former or balanced. Most of the work produced to date on asteroid capture and de-spin envisage a scenario where the mass and size of the asteroid and the spacecraft are of the same order of magnitude or in favour of the spacecraft. Scaling up for asteroid with diameter of the order of 100m or more seems rather impractical. Finally most proposed solutions for asteroid capture and de-spin put forward different ways of enveloping the asteroid relying on friction to remove rotational energy from the asteroid. This implies collisions and mechanical designs which put at risk other aspects of the spacecraft such as appendages and other aspects of the mission and orientation to light to provide power. These studies also neglect other significant physical effects such as solar wind which perturb the attitude and spin rate further. Drawing on the fact that, according to ~\cite{BurnsJoseph1973}, “An asteroid which does not undergo collisions will ultimately spin about its major principal axis because this is the minimum energy state for rotation with conservation of angular momentum; any loss of energy causes the body to move closer to this state”, as well as on the idea that enveloping an asteroid can achieve such a result, the tentacles approach proposed for active debris removal could be expended to a more active approach. This more active approach would replace passive tentacles with chains of modular self-reconfigurable robots which would provide an overdextrous end-effector on which another spacecraft could later dock. This approach would make use of energy transfer through mass mouvement between the asteroid and the robot to eventually stabilise the spin of the overall system.
\paragraph{}The next section will examine some features of modular self-reconfigurable robots relevant to our proposed study.

\subsection{Reconfigurable robots}
\paragraph{}Self-reconfigurable robots are modular robots which are able to dynamically (i.e. while active) change shape by themselves. Their modules generally independently encapsulate all the sensors, actuators, processing power and communication tools required to perform their different functionalities. The shape shifting process can involve sequences of disconnections, connections and moves along the entire structure. They possess the potential advantages of a high degree of redundancy and therefore robustness to module failure, of versatility with many different shapes for one robot and possible combinations with other robots, of adaptability to various tasks and of cheapness of production, similar in spirit to swarm robotics, of many small and relatively non-complex  units ~\cite{StoyBook}.
\paragraph{}In ~\cite{StoyBook} is provided a useful classification of self-reconfigurable robots in terms of size and number of modules. The classification is broken down into three categories:
\subsubsection{Pack robots}
\paragraph{}These robots are formed by a number of modules of the order of several tens of modules where each module could be a functional unit. These modules have a strength of the same order as of a group of modules, enabling them to not only lift themselves up but a group several other modules as well. In these robots, each module is essential to the overall functioning of the robot imposing strict coordination between modules. Pack robots have a limited scalability (when considering hundreds of modules) due to the fact that they are either controlled centrally or their modules are tightly coupled with local communication. The expected applications for this type of robots range from exploratory/inspection type of tasks with the ability to perform gaits such as running, climbing or rolling...
\subsubsection{Herd robots}
\paragraph{}These robots are formed by a number of modules of the order of several hundreds of modules which,  taken individually, are not encapsulated functional units. Moreover their strength is limited to the order of magnitude required to lift oneself up. In order to obtain useful functionalities, the grouping of modules into functional units is required. Although herd robots architecture still require a hierarchical control approach for both hardware and software, the number of modules allows for enough redundancy as to relax the coordination requirement between modules without impacting the performance of the robot. It is probably the most challenging type of robot in terms of control with a core critical set of modules to maintain in tight coordination while allowing the rest of the modules to operate with less stringent coordination in a more swarm-like fashion. This increase in complexity and number modules allows for more functionalities to be implemented by differentiation of modules design. Therefore, herd robots present the most interesting range of potential applications for our purpose in terms of interactions with heavy objects or the building or reinforcement of structures. 
\subsubsection{Swarm robots}
\paragraph{}These robots are formed by thousands of modules with limited individual importance relative to the whole robot. Individual modules have weak functionalities despite being highly autonomous. This implies that building a functional unit requires a large number of modules. Control is decentralised and the robot develops according to local rules of interactions. As it depends on stochastic randomness, this approach is difficult to scale down to a few hundreds of modules. The main potential application for swarm reconfigurable robotics is as a type of construction materials.
\subsubsection{Desired properties of SR robots}
\begin{description}
	\item[Versatility]{is the ability to adapt or be adapted to many different functions or activities.}
	\item[Adaptability]{is the ability to perform tasks even if the task or the environment changes a little.}
	\item[Robustness]{is the  ability to operate for many hours and to handle hardware and software failures.}
	\item[Polymorphy]{is the ability to assume many different shapes.}
	\item[Metamorphy]{is the ability to autonomously change among different shapes.}
	\item[Scalability]{is the maintenance of the performance of the robot with an increasing number of modules.}
	\item[Responsiveness]{is the reaction/response time of the robot.}
	\item[Functionality]{is the measures how functional requirements are met.}
\end{description}
\subsubsection{Types of SR robots}
\paragraph{}There is tight coupling between hardware and software design for specific solutions. Here are the classes of SR robots which are relevant for design:
\begin{description}
	\item[Chain-type]{which are chains of modules primarily design for fixed shape locomotion.}
	\item[Lattice-type]{where modules are positioned in a lattice structure like atoms in a crystalline solid. They are easy to reconfigure but perform efficient gait with difficulty.}
	\item[Hybrid]{which could be both in lattice as well as in a tree or chain topology. They can exit in two forms: either  chain-type or  lattice-type but not at the same time. These robots will often use a chain-type form to achieve efficient locomotion then change into a lattice-type for self-reconfiguration.}
\end{description}
\subsubsection{Computing and Communication Infrastructure}
\paragraph{}It is possible to run small operating systems on modules and use convenient programming abstractions such as threads. However, the development of controllers has hardly been explored so far. 
\paragraph{}Communications can be centralised. This is when actuators of all modules connect to a centralised host computer. It can be local between neighbouring modules or global over the entire structure and even multi-modal by combining both local and global communication. Finally, communication can be stigmergic when the environment is used as a medium.
\paragraph{}Finally, sensing has so far hardly been developed on SR robot and is still very limited.
\subsubsection{The Self-Reconfiguration problem}
\paragraph{}Self-reconfigurable robots have modules which can move around w.r.t each other to change the overall shape of the robot. The self-reconfiguration problem deals with how to move these modules around to facilitate a useful change of shape. Usually software solutions are hardware and application specific.
\paragraph{}The self-reconfiguration problem can be broken down into sub-classes of problems depending on the angle of attack to solving the problem.  Self-Reconfiguration can be thought of as search. Given an initial configuration and a goal configuration, it seeks to find a sequence of module moves that will reconfigure the robot from the initial configuration to the goal configuration.  Self-Reconfiguration can be thought of as the development of a controller that makes individual modules move in such a way that the reconfigurable robot as a whole will change from a given initial configuration to a given goal configuration. Finally, self-Reconfiguration can be thought of as being task-driven. The developed controllers aim to make individual modules move in such a way that the reconfigurable robot as a whole performs its task. In this case, the configuration emerges as secondary effect. In practice useful solutions incorporate elements of all 3 formulations. First, the goal configuration is derived from the task, second modules are controlled distributively and at the same time some search techniques are used.
\paragraph{}The self-reconfiguration problem is difficult	 and so far only specific solutions exist. Simplification can be made, but the difficulty essentially stems from several factors related to the motion constraints of the modules and their connectivity, the type of  subconfigurations dealt with, whether or not there are configurations with local minima and whether modules get in each other's way.
\subsubsection{Task-driven self-reconfiguration the way forward}
\paragraph{}From many applications cannot be inferred a desired goal configuration. If the task environment is not known beforehand, is dynamic or is complicated, determining a goal configuration may prove intractable. A task-driven approach is therefore appealing because it would likely be easier to design and implement. Since what matters is the task to be performed, constraints on  the self-reconfiguration process can be relaxed to not become the focus of the design process and interfere with the task. This is approach that will guide the design of our robot.
\subsubsection{Manipulations and Gait}
\paragraph{}The robot we intend to design will essentially perform a manipulation task in fixed or dynamic configuration. It is in this respect interesting to mention how robotic gaits are controlled as some aspects are either similar or could be useful for manipulation. Moreover, there is more research available for gaits than for manipulation. Gait control is perform with fixed configurations.
\paragraph{}For locomotion, it is desirable to have a locomotion type where momentum can be transferred and does not in involve self-reconfiguration (which is what most animals do). Locomotion is cyclic which is a feature readily exploited by control methods. Caterpillaring, side-winding, walking using 4 6 legs, rolling, climbing, tight rope walking have all already been demonstrated. Below is a list of current gait control methods.
\paragraph{}Gait control tables represent complete cycle of gait along with synchronisation. They are most easily implemented using a centralised controller.
\paragraph{}Hormone-based control introduces delay in the sequences of steps. The sequence of one module is delayed a fixed number of steps compared with a neighbouring module. This approach has two advantages, first, modules stay synchronised and second, modules can added or removed at will.
\paragraph{}Role-based control introduces the idea of capturing gait as one cyclic motion and not a discrete set of actions separated by coordination pauses.
\paragraph{}The control of complex gait can be distributed in order to deal with the fact that modules perform different motions, depending on their position in the configuration. It usually works with modules selecting their function based on the local configuration and/or their parent's function if that is not enough.
\paragraph{}The gait control method introduced above are good for hard coding. But to date there is no real method to automate development of gaits. A promising approach which has also potential application to manipulation is general pattern generation (CPG). It is a bioinspired approach based on neural networks and produces a systematic output similar to the action function in role-based control. Networks can be evolved by genetic algorithms to control the gait of a SR robot.  A general pattern generator can sense a change in the terrain through a change in gait and use this information to automatically optimise the gait online.
\paragraph{}Manipulations with self-reconfigurable robots are still very much work in progress. If modules are connected in a chain configuration, they can form a serial manipulator with properties not unlike those of traditional robot manipulators. But there to get there they are two problems to solve: how to calculate the inverse kinematics for a chain of modules and how to increase strength of a modular manipulator since it is relatively weak limited actuator strength of individual module. The challenge of this study will be to find a way to make modules produce cooperative actuations that allows them to produce larger forces that those produced individually.
\subsubsection{Interesting research challenges}
\paragraph{}The complexity of real tasks is currently met by a lack of understanding on how to develop intelligent controllers. Since SR robots have a flexible body structure, controller and bodily structure can concurrently be optimised. This is important because the behaviour of a robot and thus its ability to perform a task is the result of an interaction  among environment body and brain. SR robots can optimise their bodily structure to make every task their encounter as easy as possible to perform allowing for wider ranges of tasks while avoiding a simplicity v.s. versatility trade-off. It is also possible to envisage the performance of tasks that are too complex for conventional robots because the bodily structures of conventional robots are less optimised for tasks that do not land themselves to solutions based on a robot arm.
\subsubsection{From Basic Functionalities to Behaviours}
\paragraph{}In order to tackle the complexity of real tasks basic functionalities need to be improved and then combined into a coherent controller. Behaviour based controllers are composed of a number of behaviours each focusing on solving one aspect of a task (ex obstacle avoidance, navigation...). Behaviours do not prescribe a way of implementing themselves and any method can do. Behaviours provide a way of simplifying tasks to make them manageable rather than simplifying the task itself by spiting them into small well-defined subtasks. A behaviour based approach allows for looking at different aspects of a real task in isolation.
\paragraph{Behaviours Adaptation}Behaviours can be optimised for a specific task. Behaviour adaptation is characterised by not changing the behaviour radically. It allows for the robot to adapt and optimise its behaviour for aspects of a task that were not known at the time of implementation. They are but a few studies of behaviour adaptation of SR robots in the literature but they tend to either adapt the controller or the body but not both at the same time and only in the context of stereotypical tasks. One major problem which remains open is the implementation of online adaptation.
\paragraph{Behaviour Selection}Behaviour selection is the process of choosing which behaviour is active. It is a continuous process that monitors sensors and even the internals of some behaviours to decide which behaviour should be active and can decide to replace some behaviours with others while operating and completely change the behaviour mode of the robot.
\paragraph{Behaviour Mode}Behaviour mode control is a set of behaviours that together with appropriate behaviour selection make the robot perform a task in a specific way. In addition to these behaviour modes, transition modes are required to allow the robot to change from one behaviour mode to another. In the most general form, a controller is a network of behaviour modes potentially with transition modes connecting them. Individual behaviours may also be reused in different behaviour modes if it makes sense. Behaviour mode control provides a unedified way to look at control of SR robots. Real tasks can be divided into simpler subtasks. Optimal behaviours can be designed both in terms of control and bodily structure for each of these tasks. Behaviours can be combined to form a complete solution to the task and even let them, through behaviour adaptation, adapt to  changes in the environment or task and thus provide a robot  with adaptability and robustness. Behaviour mode selection  can also completely change the behaviour of a robot and thus demonstrate its versatility.

\section{Dissertation outline}
\paragraph{Chapter 1: Introduction}
A brief introduction of the main topic and the motivation behind it addressing the following questions: 
\paragraph{}	What is the problem?
\paragraph{}	Why is the problem important?
\paragraph{}	What has so far be done on the problem?
\paragraph{}	What is the contribution of the thesis on the problem?
\paragraph{}	Why is the contribution original?
\paragraph{}	Why is the contribution non-trivial?
\paragraph{}Finally a synopsis of the subsequent sections according to the content of the subsequent chapters.
%\paragraph{Chapter 2: Literature Review}